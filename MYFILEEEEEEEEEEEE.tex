\documentclass[10pt]{report}
\usepackage{multirow}
\usepackage{array}
\usepackage{color}

\newcommand{\blueIt}[1]{\itshape \textcolor{blue}{#1}}
\setcounter{section}{0}

\setcounter{dbltopnumber}{3} % sets the top table number allowed

\begin{document}

\title{\bfseries \Huge A report on \\
\blueIt{``Elements of Styles"}}

\author{\bfseries Muhtasim Ulfat Tanmoy\\
1405086}
\date{\today}

\maketitle

\newpage

~

\newpage



\tableofcontents


\chapter* {Preface}

This report gives a summary of the book ‘The Elements of Style’.\\
This is an assignment for the course CSE300 : Technical 
Writing \& Presentation and it is
written using \LaTeX\\

It discusses the principal requirements of plain English style and concentrates attention on the rules
of usage and principles of composition most commonly violated. 


\addcontentsline{toc}{chapter}{Preface}'



\chapter {INTRODUCTORY}
The Element of Style written by William Strunk Jr. and edited by E.B. White.\\
This is a simple report on that book covering all the basic usage and good practices.\\
This report covering those rules has the following chapters:\\
{\blueIt{Chapter 2}} tells us about Elementary Rules of Usage.\\
{\blueIt{Chapter 3}} goes into the detail about Elementary Principles of Composition\\
{\blueIt{Chapter 4}} tells us about A Few Matters of Form\\
{\blueIt{Chapter 5}} gives overview about Words and Expressions Commonly Misused\\
{\blueIt{Chapter 6}} shows us Words often misspelled\\
{\blueIt{Chapter 7}} shows us implementation of first lecture\\



\chapter {ELEMENTARY RULES OF USAGE}

\section {\bfseries \ Form the possessive singular of noun with ’s}\label{sec:one}

It is good way to form possessive singular of nouns by adding ’s.

Rule for final consnent:

\begin{table}[!tbh]
    \centering
        \begin{tabular}{|c|}
        \hline
        Mimon’s friend\\
        \hline
        Sadman’s poems\\
        \hline
        his friend’s malice\\   
        \hline
        \end{tabular}
    
\end{table}
\noindent
Exceptions are the possessives of ancient proper names in -es and -is, the possessive
Jesus’.

\begin{table}[!tbh]
    \centering
        \begin{tabular}{|c|}
        \hline
        the heel of Achilles\\
        \hline
		the laws of Moses\\
		\hline
		the temple of Isis\\
		\hline
        \end{tabular}
    
\end{table}

\noindent
The pronominal possessives hers, its, theirs, yours, and oneself have no apostrophe

\pagebreak

\section {\bfseries \ In a series of three or more terms with a single conjunction, use a
comma after each term except the last}\label{sec:two}

When we use single conjunction such that and , or etc to make a series of making
two or more terms , we should use comma after every term except the last.
Thus write,\\

\begin{table}[!tbh]
    \centering
        \begin{tabular}{|c|}
        \hline
        red, white, and blue\\
        \hline
		good, honest, but lajy\\
		\hline
		Rashed closed the door, run away and made a mess of it\\
		\hline
        \end{tabular}
    
\end{table}

\noindent
In the names of business firms the last comma is omitted, as

\begin{table}[!tbh]
    \centering
        \begin{tabular}{|c|}
        \hline
        Reve, ipay and Kona\\
        \hline
        \end{tabular}
    
\end{table}

The abbreviation etc., even if only a single term comes before it, is always preceded by
a comma.


\section {\bfseries \ Enclose parenthetic expressions between commas}\label{sec:three}

\begin{table}[!tbh]
    \centering
        \begin{tabular}{|l|}
        \hline
        The best way to complete an assignment, unless you are pressed for\\
time, is to explore all the details and understand the full concept.\\
        \hline
        \end{tabular}
    
\end{table}
\noindent
This rull is quite difficult for application.Below we have cases that where the comma must not be ommited:

\begin{table}[!tbh]
    \centering
        \begin{tabular}{|c|}
        \hline
        My friend, Maksudul Alam paid us a visit\\
yesterday.\\
        \hline
        My friend you will be pleased to hear, is now getting\\
great at sports.\\
		\hline
	  
        \end{tabular}
    
\end{table}


\pagebreak
\noindent
Non-restrictive relative clauses are, in accordance with this rule, set off by commas.


\begin{table}[!tbh]
    \centering
        \begin{tabular}{|l|}
        \hline
        The players, who had at first been lazy, became\\
more and more skilled.\\
        \hline
        
	  
        \end{tabular}
    
\end{table}
\noindent
Where and when are similarly punctuated.

\begin{table}[!tbh]
    \centering
        \begin{tabular}{|l|}
        \hline
        In 1971, when Bangladesh was born, there was so many happiness amog people. \\
        \hline
       
	  
        \end{tabular}
    
\end{table}


\section {\bfseries \ Place a comma before and or but introducing an independent
clause}\label{sec:four}

\begin{table}[!tbh]
    \centering
        \begin{tabular}{|l|}
        \hline
      The early years of our life have disappeared, and we can never get that back\\
        \hline
        The environment is bad, but we can still go there.\\
\hline

 \end{tabular}
 \end{table}

This sentences may be thought of to be rewritten.Like this

\begin{table}[!tbh]
    \centering
        \begin{tabular}{|l|}
        \hline
      The early years of our life have disappeared, we can never get that back\\
        \hline.
        The environment is bad, we can still go there.\\
\hline

 \end{tabular}
 \end{table}
 
 \pagebreak
 
 
but we can also use no comma is needed after the conjunction.

 \begin{table}[!tbh]
    \centering
        \begin{tabular}{|l|}
        \hline
     The environment is bad,but if we can stick together, we can still go there.\\
        \hline
        

 \end{tabular}
 \end{table}  
	
	
	\section {\bfseries \ Do not join independent clauses by a comma}
	
	For compound sentence semicolon can be used,

\begin{table}[!tbh]
    \centering
        \begin{tabular}{|l|}
        \hline
     We are going to rangamati;it is a beautiful place.\\
        \hline
        We are late;we cant go there.
dark.\\
\hline
        

 \end{tabular}
 \end{table}




We can use period in stead of semicolon.
\begin{table}[!tbh]
    \centering
        \begin{tabular}{|l|}
        \hline
     We are going to rangamati.It is a beautiful place.\\
        \hline
        It is nearly half past five. We cannot reach town
before dark.\\
\hline
        

 \end{tabular}
 \end{table}
 
 \pagebreak
 
 This can be used in conjunction,
 \begin{table}[!tbh]
    \centering
        \begin{tabular}{|l|}
        \hline
     The weather is bad,but if we act now,we can go there.\\ 
        \hline
        

 \end{tabular}
 \end{table}
        
	For two-part sentences connected by an adverb, see the next section.  
	
	
	\section {\bfseries \ Do not join independent clauses by a comma} 
 \label{sec:six}
	
	if there are two or more grammatically complete clause and not joined , they
should be joined using semicolon.

\begin{table}[!tbh]
    \centering
        \begin{tabular}{|l|}
        \hline
    It has been raining with thunder since morning;you shouldn’t go out
for marketing.\\
        \hline

        

 \end{tabular}
 \end{table}

But using period instead of semicolon is also correct.
\begin{table}[!tbh]
    \centering
        \begin{tabular}{|l|}
        \hline
     It has been raining with thunder since morning.You shouldn’t go out
for university.\\
\hline
        

 \end{tabular}
 \end{table}

 thus
 
 \begin{table}[!tbh]
    \centering
        \begin{tabular}{|l|}
        \hline
    It is nearly half past five, and we cannot reach town before
dark.\\
        \hline
        
      
        

 \end{tabular}
 \end{table}
 
   semicolon needed if preceeded by adverb.
 
 
 \begin{table}[!tbh]
    \centering
        \begin{tabular}{|l|}
        \hline
     I had never been in the place before; so I had difficulty in
finding my way about.\\
        \hline
        

 \end{tabular}
 \end{table}


thus :
 
 
 
 \begin{table}[!tbh]
    \centering
        \begin{tabular}{|l|}
        \hline
    As I had never been in the place before, I had difficulty in
finding my way about.\\
        \hline
        

 \end{tabular}
 \end{table}
 
 For very short sentence use comma:

 
 
 \begin{table}[!tbh]
    \centering
        \begin{tabular}{|l|}
        \hline
        I woke up,brushed and left for university\\
\hline
        

 \end{tabular}
 \end{table}
 
 
 \section {\bfseries \ Do not break sentences in two}
 
 In other words, do not use periods for commas.
 
  \begin{table}[!tbh]
    \centering
        \begin{tabular}{|l|}
        \hline
     I met him on buet several years ago. Coming
home from my hometown.\\
       
\hline
        

 \end{tabular}
 \end{table}
 
 It should be replaced by comma.

\pagebreak

It is permissible to make an emphatic word or expression serve the purpose of a sentence
and to punctuate it accordingly:

\begin{table}[!tbh]
    \centering
        \begin{tabular}{|l|}
        \hline
     Again and again he called out. No reply.\\
        \hline
       
        

 \end{tabular}
 \end{table}
 
 \section {\bfseries \ Do not break sentences in two}


\begin{table}[!tbh]
    \centering
        \begin{tabular}{|l|}
        \hline
    Walking slowly down the road on buet, i saw a woman accompanied
by two children.\\
        \hline
       
        

 \end{tabular}
 \end{table}
 
 It can be referred to the woman by:


\begin{table}[!tbh]
    \centering
        \begin{tabular}{|l|}
        \hline
    I saw a woman, accompanied by two children, walking
slowly down the road on buet.\\
        \hline
       
        

 \end{tabular}
 \end{table}
 

 
 Sentences violating this rule:
 
 \begin{table}[!tbh]
    \centering
        \begin{tabular}{|l|}
        \hline
    Being in a hurried condition, he was able to get the
house very quickly.\\
        \hline
       
        

 \end{tabular}
 \end{table}
 
 
 
 \section {\bfseries \ Divide words at line-ends, in accordance with their formation and
pronunciation}

If there is room at the end of a line for one or more syllables of a word, but not for
the whole word, divide the word:\\


A. Divide the word according to its formation:
 \begin{table}[!tbh]
    \centering
        \begin{tabular}{|l|}
        \hline
    know-ledge (not knowl-edge)\\
        \hline
       
        
     
     
     
       
        

 \end{tabular}
 \end{table}
       
       
       B. Divide “on the vowel:”
  
  \begin{table}[!tbh]
    \centering
        \begin{tabular}{|l|l|}
       \hline edi-ble (not ed-ible) & propo-sition\\
       \hline
       
ordi-nary & espe-cial\\
\hline
reli-gious & oppo-nents\\
\hline

       
        

 \end{tabular}
 \end{table} 
 
 
 C. Divide between double letters: \\

\begin{table}[!tbh]
    \centering
        \begin{tabular}{|l|l|}
     \hline Apen-nines & Cincin-nati\\
    \hline 

       
        

 \end{tabular}
 \end{table} 

The treatment of consonants in combination is best shown from examples:

\newpage

\begin{table}[!tbh]
    \centering
        \begin{tabular}{|l|l|}
       \hline for-tune & pic-ture\\ \hline
presump-tuous & illus-tration \\
\hline

sub-stan-tial (either division)&  indus-try\\
\hline
instruc-tion & sug-ges-tion\\ \hline
incen-diary & \\
\hline


       
        

 \end{tabular}
 \end{table} 
 
 That's how they are written.
 
 
 \pagebreak
 
 \newpage
 ~
 
 \pagebreak
 
 \chapter{ELEMENTARY PRINCIPLES OF
COMPOSITION}
 
 \section {\bfseries \ Make the paragraph the unit of composition: one paragraph to
each topic}


A paragraph should be brief.One represting one topqic:\\

\noindent
A. Account of the work.\\
B. Critical discussion.

\pagebreak

\noindent
A report on a poem, written for a class in literature, might consist of seven paragraphs:\\

\noindent
A. Facts of composition and publication.\\
B. Kind of poem; metrical form.\\
C. Subject.\\
D. Treatment of subject.\\
E. For what chiefly remarkable.\\
F. Wherein characteristic of the writer.\\
G. Relationship to other works.\\

\noindent
A novel might be discussed under the heads:\\


A. Setting.\\
B. Plot.\\
C. Characters.\\
D. Purpose.\\


A historical event might be discussed under the heads:\\


A. What led up to the event.\\
B. Account of the event.\\
C. What the event led up to.\\

This is best learned from examples in well-printed
works of fiction.

 \section {\bfseries \ As a rule, begin each paragraph with a topic sentence; end it in
conformity with the beginning}

Understanding the topic in paragraph:\\
\noindent
A. the topic sentence comes at or near the beginning;\\
B. the succeeding sentences explain or establish or develop the statement
made in the topic sentence; and\\
C. the final sentence either emphasizes the thought of the topic sentence or
states some important consequence\\



In a long paragraph, he may carry out several of these processes.\\


\newpage





\section {\bfseries \ . Use the active voice}

The active voice is usually more direct and vigorous than the passive:

\begin{table}[!tbh]
    \centering
        \begin{tabular}{|l|}
        \hline
       My first visit to Buet will always be remembered
by me.\\
       \hline


       
        

 \end{tabular}
 \end{table} 
 
 This is much better than
 
\begin{table}[!tbh]
    \centering
        \begin{tabular}{|l|}
        \hline
       My first visit to Buet will always be remembered,\\
       \hline


       
        

 \end{tabular}
 \end{table} 
 
 The latter sentence is less direct, less bold, and less concise. If the writer tries to make
it more concise by omitting “by me,”
 
\begin{table}[!tbh]
    \centering
        \begin{tabular}{|l|}
        \hline
       I shall always remember my first visit to Buet.\\
       \hline


       
        

 \end{tabular}
 \end{table} 
 
 
 
 Passive voice should be discarded.



 
 
And these examples shows how they can work.


\begin{table}[!tbh]
    \centering
        \begin{tabular}{|m{18em}|m{18em}|}
        \hline
      There were a great number
of dead leaves lying
on the ground.&
Dead leaves covered
the ground.\\ \hline
The sound of the falls could
still be heard.&
The sound of the falls still
reached our ears.\\
\hline

       
       


       
        

 \end{tabular}
 \end{table}
 
 
 \section {\bfseries Put statements in positive form}

 	\begin{table}[!tbh]
    \centering
        \begin{tabular}{|m{18em}|m{18em}|}
        \hline
 He was not very often on time. He usually came late.&
He did not think that studying
Bangla was much use.\\
\hline
    

 \end{tabular}
 \end{table}
 
 
 
 The last example:

	\begin{table}[!tbh]
    \centering
        \begin{tabular}{|m{10em}|m{10em}|}
        \hline

not honest & dishonest\\
\hline
did not remember & forgot \\ 
\hline
did not pay any attention to & ignored\\
\hline


       
        

 \end{tabular}
 \end{table}
 
 
 
 \section {\bfseries Omit needless words}
 
 No need of extra words\\


Many expressions in common use violate this principle:


\begin{table}[!tbh]
    \centering
        \begin{tabular}{|m{10em}|m{10em}|}
        \hline

he is a man who & he\\
\hline
this is a subject which this & subject\\
\hline
His story is a strange one. & His story is strange.\\ \hline




       
        

 \end{tabular}
 \end{table}
 
 
 \section {\bfseries Avoid a succession of loose sentences}
 
 This rule refers especially to loose sentences of a particular type.
 
 Only unskilled writer would do that.Should be avoided.

\section {\bfseries Exprtess co-ordinate ideas in similar form} \label {sec:seven}


Parallel ideas should be written in parallel form , otherwise it makes readers
irrelevant thoughts.





\begin{table}[!tbh]
    \centering
        \begin{tabular}{|m{18em}|m{18em}|}
        \hline
	Make your path and go a long run.\\
\hline


       
        

 \end{tabular}
 \end{table}
 



By this principle, an article or a preposition applying to all the members of a series
must either be used only before the first term or else be repeated before each term.


\begin{table}[!tbh]
    \centering
        \begin{tabular}{|m{10em}|m{10em}|}
        \hline

		The French, the Italians, Spanish,
and Portuguese &
The French, the Italians, the
panish, and the Portuguese\\
\hline


       
        

 \end{tabular}
 \end{table}
 







       
 
 
 Modifiers should come, if possible next to the word they modify.Cant be modified by more modifiers.\\

\noindent\rule[0.5ex]{\linewidth}{2pt} %newRule from net


\newpage

 \section {\bfseries  In summaries, keep to one tense}
 
 
One tense for full summary.

	\begin{table}[!tbh]
    \centering
        \begin{tabular}{|m{18em}|m{18em}|}
        \hline
        
        An unforeseen chance prevents Friar John from delivering
Friar Lawrence’s letter to Romeo. Juliet, meanwhile,
owing to her father’s arbitrary change of the day set for
her wedding, has been compelled to drink the potion
on Tuesday night, with the result that Balthasar informs
Romeo of her supposed death before Friar Lawrence
learns of the nondelivery of the letter.\\
\hline



       
        

 \end{tabular}
 \end{table}
 

 
 



 \section {\bfseries Place the emphatic words of a sentence at the end}
 
 \begin{table}[!tbh]
    \centering
        \begin{tabular}{|m{18em}|m{18em}|}
        \hline
        Humanity has hardly advanced
in fortitude since that time,
though it has advanced in many
other ways.&
Humanity, since that time, has
advanced in many other ways,
but it has hardly advanced
in fortitude.\\
\hline
This steel is principally used
for making razors, because of
its hardness.&
Because of its hardness, this
steel is principally used in making
razors.\\
\hline
 \end{tabular}
 \end{table}
 
 



 
Anything that goes at first becomes emphatic.

\

 

 


\newpage

~

\newpage

\chapter {A FEW MATTERS OF FORM}

\begin{itemize}
  \item Headings. Leave a blank line, or its equivalent in space, after the title or heading. On later pages, if using ruled paper, begin on the first
line.
  \item Numerals. Do not spell out dates or other serial numbers. Write them in figures
or in Roman notation, as may be appropriate.
	\begin{table}[!tbh]
    \centering
        \begin{tabular}{|m{10em}|m{10em}|}
        \hline
        August 9, 1918 & Chapter XII\\
        \hline

 \end{tabular}
 \end{table}
 
 \item 
Parentheses. A sentence containing an expression in parenthesis is punctuated,
outside of the marks of parenthesis, exactly as if the expression in parenthesis
were absent. 
\begin{table}[!tbh]
    \centering
        \begin{tabular}{|m{18em}|}
        \hline
        I went to his house yesterday (my third attempt to see
him), but he had left town.\\
\hline

 \end{tabular}
 \end{table}




\item
References. In scholarly work requiring exact references, abbreviate titles that
occur frequently, giving the full forms in an alphabetical list at the end.

 \begin{table}[!tbh]
    \centering
        \begin{tabular}{|m{18em}|m{18em}|}
        \hline
        In the second scene of the
third act &
In III.ii (still better, simply insert
III.ii in parenthesis at the
proper place in the sentence)\\
\hline
\multicolumn{2}{|c|}{After the killing of Polonius, Hamlet is placed under guard
(IV. ii. 14).}\\
\hline
2 Samuel i:17-27 & Othello II.iii. 264-267,
III.iii. 155-161 \\
\hline
 \end{tabular}
 \end{table}volume, page, except
when referring by only one of them. Punctuate as indicated below.

\item 
Titles. For the titles of literary works, scholarly usage prefers italics with capitalized
initials. 

\begin{table}[!tbh]
    \centering
        \begin{tabular}{|m{18em}|m{18em}|}
        \hline
        The Iliad; the Odyssey; As You Like It; To a Skylark;
The Newcomes; A Tale of Two Cities; Dickens’s Tale of
Two Cities\\
\hline
 \end{tabular}
 \end{table}




\end{itemize}

\newpage



~

\newpage



\chapter{WORDS AND EXPRESSIONS
COMMONLY MISUSED}


{\bfseries Dependable.} A needless substitute for reliable, trustworthy.\\


\noindent
{\bfseries State.} Not to be used as a mere substitute for say, remark. Restrict it to the
sense of express fully or clearly, as, He refused to state his objections.\\

\noindent
{\bfseries Very.} Use this word sparingly. Where emphasis is necessary, use words strong
in themselves.\\



\noindent
{\bfseries Etc.} Not to be used of persons. Equivalent to and the rest, and so forth, and hence
not to be used if one of these would be insufficient, that is, if the reader would
be left in doubt as to any important particulars. Least open to objection when it
represents the last terms of a list already given in full, or immaterial words at the
end of a quotation.
At the end of a list introduced by such as, for example, or any similar expression,
etc. is incorrect.\\

\noindent
{\bfseries Fact.} Use this word only of matters of a kind capable of direct verification, not of
matters of judgment. That a particular event happened on a given date, that lead
melts at a certain temperature, are facts. But such conclusions as that Napoleon
was the greatest of modern generals, or that the climate of California is delightful,
however incontestable they may be, are not properly facts.
On the formula the fact that, see under Rule \ref{sec:three}\\

\noindent
{\bfseries Less.} Should not be misused for fewer.
\begin{table}[!tbh]
    \centering
        \begin{tabular}{|m{18em}|m{18em}|}
        \hline
        He had less men than in the previous
campaign. &
He had fewer men than in the
previous campaign.\\
\hline
 \end{tabular}
 \end{table}
 
\noindent
Less refers to quantity, fewer to number. “His troubles are less than mine” means
“His troubles are not so great as mine.” “His troubles are fewer than mine”
means “His troubles are not so numerous as mine.” It is, however, correct to say,
“The signers of the petition were less than a hundred, “where the round number,
a hundred, is something like a collective noun, and less is thought of as meaning
a less quantity or amount.\\


\noindent
{\bfseries Compare.} To compare to is to point out or imply resemblances, between objects regarded
as essentially of different order; to compare with is mainly to point out
differences, between objects regarded as essentially of the same order. Thus
life has been compared to a pilgrimage, to a drama, to a battle; Congress may
be compared with the British Parliament. Paris has been compared to ancient
Athens; it may be compared with modern London.\\

\noindent
{\bfseries Clever.} This word has been greatly overused; it is best restricted to ingenuity displayed
in small matters\\

\newpage
\chapter {WORDS OFTEN MISSPELLED}

\begin{table}[!tbh]
    \centering
        \begin{tabular}{m{10em} m{10em} m{10em}}
        
        accidentally & formerly & privilege\\
advice &humorous &pursue\\
affect &hypocrisy &repetition\\
beginning &immediately &rhyme\\
believe &incidentally &rhythm\\
benefit &latter &ridiculous\\
challenge &led &sacrilegious\\
criticize &lose &seize\\
deceive &marriage &separate\\
definite &mischief& shepherd\\
describe &murmur &siege\\
despise &necessary &similar\\
develop &occurred &simile\\
disappoint &parallel &too\\
duel &Philip &tragedy\\
ecstasy &playwright& tries\\
effect &preceding &undoubtedly\\
existence &prejudice &until\\
fiery &principal &\\
 \end{tabular}
 \end{table}
 
 \noindent
 Write to-day, to-night, to-morrow (but not together) with hyphen.\\
 
 \noindent
Write any one, every one, some one, some time (except the sense of formerly) as two
words.

\chapter {1st Lecture}


\noindent
{\bfseries \Large All commands from 1st lecture.}\\


\noindent
{\Large Text:}{\newline}
Space separation ~~~~~~~~~~~~ Manually giving space with tilde.\\


\noindent
{\Large Symbols:}
\%
\\

\noindent
{\Large Textmode:}\\
\textbf{Bold}\\
\textit{Italic}\\
\textsc{Another text}\\
\textsf{Another text}\\
\emph{Another text}\\
{\bfseries Bold font}\\
{\itshape Italic }\\
{\em Emphasized}\\

\noindent
{\Large Color:}\\
\textcolor{red}{red colored text}\\




\noindent
{\Large List:}\\
\begin{enumerate} %ordered list
 \item A
 \item B
 \end{enumerate}

 \begin{itemize} %unordered list
 \item A
 \item B
 \end{itemize}

 \begin{description} %description list
 \item[A] Some details
 \end{description}
 
 
 \center
{ \bfseries THE END}


\chapter {Conclusion}

\noindent
Rules described in this report may help the writers a lot.
These rules will make their writings
more appreciable, understandable, and attractive. They can also avoid common mistakes.



Writers should be careful about their writing styles as it is very important for
attracting readers.
If writer fails to make understand the readers his words
properly then it may become useless. 
This was fully done for an assignment and the contents are copied from the
book Elements of Styles


\end{document}
