\documentclass{article}
\usepackage{multirow}
\usepackage{amsmath}
\usepackage{graphicx}
\usepackage[letterpaper, total={6in, 8in}]{geometry}

\begin{document}
\begin{twocolumn}

\title{Online on \LaTeX~Lecture 2 \& 3}
\author{Azmain Adel\\1405075}
\maketitle

Today’s online is\footnote{A sample footnote} about implementing \textbf{Tables, Figures, Bibliography} and some \textbf{Mathematical Notations} as we have been 
practising for the last two lectures. We will create a table with multiple
 rows and columns in Section \ref{sec:1}, a figure in Section \ref{sec2}, a bibliography 
 in Section \ref{sec3} and experiment with some mathematical symbols in Section \ref{sec4}.

\section{Table} \label{sec:1}
\textbf{Table} \ref{table1} contains multiple rows and columns. Do not forget to use proper package to get this done.

\begin{table}[!tbh]
  \centering
    \begin{tabular}{|c|c|c|}
      \hline
      \multicolumn{3}{|l|}{\textbf{Header}}\\
      \hline
      \textbf{Column1} & \multicolumn{2}{|c|}{\textbf{Column2}}\\
      \hline
      \multirow{2}{*}{Row1} & \textbf{Row1} & \textbf{Row1}\\
      \cline{2-3}
      & \textbf{Row2} & \textbf{Row2}\\
      \hline
    \end{tabular}
  \label{table1}
  \caption{\textbf{This is an example table.}}
\end{table}

\section{Figure} \label{sec2}
\textbf{Figure} \ref{fig1} displays the BUET logo. Notice that it is displayed in single column within the two column article format. Do not forget to include proper package here too.


\section{Bibliography} \label{sec3}
You need to create your own \textbf{.bib} file that contains \textbf{three bibliographic references} in the following format:
\\\\
\texttt{@book\{bib\_label,\\
title=\{The title of the Book\},\\
author=\{Name of the author\}\}
}
\\~\\
\noindent
Refer only two of them in this Section (citation \cite{eos} and).

\section{Mathematical Symbols and Equations} \label{sec4}
This is an integral $\int_a^b x^2dx$ inside text. Another way is the following:
\begin{displaymath}
  \int_a^b x^2dx
\end{displaymath}

\noindent
The binomial coefficient is defined by the next example:
\begin{equation}
  \binom{n}{k} = \frac{n!}{k!(n-k)!}
\end{equation}

\end{twocolumn}

\begin{onecolumn}

\begin{figure}[h]
  \centering
  \includegraphics[width=0.3\textwidth]{buetlogo.png}
  \caption{\textbf{BUET Logo}}
  \label{fig1}
\end{figure}

\bibliographystyle{plain}
\bibliography{yo}
\end{onecolumn}

\end{document}


%bibtex lesson7a1


//////////////////////////////



\documentclass{article}

\usepackage{amsmath}

\begin{document}
\title{\LaTeX{} Lecture 2}
\date{}
\maketitle

\section{Footnote}
This line has a footnote.\footnote{A sample footnote}

\section{Font Size}

\subsection{Predefined Sizes}
{\tiny This is in tiny fontsize.}\\
{\small This is in small fontsize.}\\
{\large This is in large fontsize.}\\
{\Large This is in Large fontsize.}\\
{\LARGE This is in LARGE fontsize.}

\subsection{Custom Size}
{
\fontsize{4cm}{1cm}\selectfont
This is ridiculously large (4cm size, 1cm line space).
}

\section{Changing Fonts}
{
	\fontencoding{T1}
	\fontfamily{phv}
	\fontseries{m}
	\fontshape{it}
	\selectfont
	This is Helvetica in medium series, italic. Check out some other options from Wikibook.\\
}
{
	\fontencoding{\encodingdefault}
	\fontfamily{\familydefault}
	\fontseries{\seriesdefault}
	\fontshape{\shapedefault}
	\selectfont
	Back to the defaults.\\
}
{
	\usefont{T1}{phv}{m}{it}Back to Helvetica, but with different syntax.
}

\section{Verbatim}
Verbatim is used when you want to output someting word for word as it is without any interpretation.\\
This is inline verbatim: \verb|!@#$%^&*|.\\~\\
This is block level verbatim:
\begin{verbatim}!@#$%^&*\end{verbatim}

\section{Mathematical Symbols}
It is better to include the \textbf{amsmath} or the \textbf{mathtools} package before writing mathematical symbols.
Many symbols will actually work without them, but some will require either one of the packages.
We discussed a handful of symbols in class, if you need to use more checkout the \LaTeX{} Wikibook.

\subsection{A Simple Example}
This is a simple equation: a + a = 2a\\
This is a simple equation: $a + a = 2a$\\
\\~\\
The equation in first line is in normal text, whereas the one in the second line is in a mathematical environment.

\subsection{Mathematical Environments}
Various mathematical environments along with their shorthand notation are presented in Table~\ref{tab:vme}.
\begin{table}[h]
	\centering
	\caption{Various Mathematical Environments}
	\label{tab:vme}
	\begin{tabular}{|l|l|l|}
		\hline
		\textbf{Environment} & Shorthand 1 & Shorthand 2\\
		\hline
		math (inline) & \$ \ldots \$ & \textbackslash ( \ldots \textbackslash )\\
		displaymath (block level) & \$\$ \ldots \$\$ & \textbackslash [ \ldots \textbackslash ]\\
 equation (block level, numbered) & ~ & ~\\
		\hline
	\end{tabular}
\end{table}

\subsection{Summation and Indices}
Inline summation: $\sum_{i=0}^na_i$\\
Inline summation with displaystyle: $\displaystyle\sum_{i=0}^na_i$\\
Summation in block level math:
$$\sum_{i=0}^na_i$$
Summation in equation:
\begin{equation}
\sum_{i=0}^na_i
\end{equation}

\subsection{Miscellaneous}
Change epsilon to phi, Phi, and varphi in the following equation observe the results.
\begin{displaymath}
	\forall x \in X, \quad \exists y \leq \epsilon
\end{displaymath}
\\~\\
Use operator commands when available, otherwise use textrm/mathrm.
\begin{displaymath}
	\cos2\theta = \cos^2\theta - \sin^2\theta
\end{displaymath}
\begin{displaymath}
	\textrm{cos}2\theta = \mathrm{cos}^2\theta - \textrm{sin}^2\theta
\end{displaymath}
\\~\\
Limits and inifinity:
\begin{displaymath}
	\lim_{x \to \infty} \exp(-x) = 0
\end{displaymath}
\\~\\
Fractions:
\begin{displaymath}
	\frac{a}{b}
\end{displaymath}
\\~\\
Binomoals:
\begin{displaymath}
	\binom{n}{k}
\end{displaymath}
\\~\\
Times:
\begin{displaymath}
	a \times b
\end{displaymath}
\\~\\
$n$th root:
\begin{displaymath}
	\sqrt[n]{a}
\end{displaymath}
\\~\\
Mods and equivalence:
\begin{displaymath}
	a \bmod b, \quad a \pmod b, \quad a \equiv b
\end{displaymath}
\\~\\
Integrals:
\begin{displaymath}
	\int_a^b
\end{displaymath}
\\~\\
Set operations:
\begin{displaymath}
	a \cap b, \quad a \cup b, \quad a \bigcap b, \quad a \bigcup b
\end{displaymath}
\\~\\
Plus minus:
\begin{displaymath}
	a \pm 5, \quad a \mp 5
\end{displaymath}

\subsection{Automatic Sizing of Parentheses/Braces/Brackets}
Compare the following two fractions:
\begin{displaymath}
	(\frac{a}{b})
\end{displaymath}
\begin{displaymath}
	\left(\frac{a}{b}\right)
\end{displaymath}

\subsection{Matrices}
Change the matrix environment to \{pbBvV\}matrix in the following equation and observe the output:
\begin{equation}
	\begin{matrix}
		1 & 0 & 0\\
		0 & 1 & 0\\
		0 & 0 & 1
	\end{matrix}
\end{equation}

\end{document}


